\chapter{関連研究}

これまでに数多くのビデオビットレート制御に関する研究が行われてきた.

帯域幅推定に着目したビデオビットレート制御の研究がある.
Thangら\cite{Thang}は,短期的な帯域幅変動に対応し,ビットレートを安定させるため,ユーザの使用帯域推定に基づく適応的リクエスト方法を提案した.この手法では,スムージング係数の調整により短期的な変動に安定して対応しつつ,大きな変化には迅速に適応可能とした.また,メタデータに品質情報を加えることで,最適な選択を行い高品質なストリーミングを安定して提供できるようにしている.
また,実際のアプリケーションでは,ユーザの使用帯域や帯域幅を正確に推定することは容易ではない.

そこで,Maoら\cite{Mao}は,ユーザの使用帯域を正確に推定することに基づいた深層強化学習レート適応アルゴリズムを提案した.ニューラルネットワークを利用して,過去のデータからビットレート選択を自動で学習し,適応することで高いQoEを実現した.しかし,大量のデータが必要である点や,様々なビデオの特性やネットワーク条件において,同様に適応できるか等の課題がある.

さらに,ユーザの体感品質であるQoEを向上させるため,QoE最大化に基づく動的ビデオビットレート制御方法が研究された.
Xuら\cite{Xu}は,ユーザのQoEをビットレート,再生バッファの飢餓確率,および連続再生時間を組み合わせたモデルで評価し,ユーザの使用帯域変動とバッファ長に基づいてビットレートを動的に調整する2つのビデオビットレート制御アルゴリズムを提案し,ストリーミングの効率と安定性を考慮し,ユーザのQoEを最大化した.
しかし,このような手法ではユーザ個人ごとに制御を行っているため,原因となっているユーザ間の相互影響を考慮できていない.

Yanagisawa\cite{kison}やYuanら\cite{motomoto}は,複数のユーザがリンクを共有する状況をゲーム理論を用いて考慮したビデオビットレート制御手法を提案している.これらの手法は,共有リンクの帯域幅を共有資源として複数のユーザが帯域を取り合う状況をゲーム理論でモデル化し,利得関数を用いてユーザ毎に最適なビデオビットレートを決定している.
しかし,各ユーザのビデオビットレートが他ユーザの帯域への影響を十分に反映できていない.

本研究では,既存手法\cite{kison}と\cite{motomoto}におけるゲーム理論を用いたレート制御手法において,複数ユーザ間の相互影響を考慮した制御関数に改善する.