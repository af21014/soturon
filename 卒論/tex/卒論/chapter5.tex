\chapter{考察}
この章では,本研究の結果と.net Winter work shop 2024での議論を踏まえて,今後の展開について考察する.

提案された利得関数は,他ユーザの帯域利用状況を考慮して,利己的な高レート要求のユーザに対するペナルティを大きくする設計となっている.この利得関数の効果は,図\ref{fig:teianritoku}を見ると,ユーザ1が6Mbpsを要求し,ユーザ2が1Mbpsを要求した場合の利得変化に大きくに表れている.既存研究\cite{kison}\cite{motomoto}による図\ref{fig:kisonritoku}では,ユーザ1が6Mbpsを要求し,ユーザ2が1Mbpsを要求した場合ユーザ1の利得が-4.5であったのに対し,提案手法では-57.0という極端に低い値をとった.この52.5の利得低下は,提案手法が利己的なユーザにペナルティを課す仕組みを反映している.また,提案手法の特徴として,ユーザごとに異なるペナルティを課す点もある.利己的なユーザにペナルティを課す一方,リンクを共有している他ユーザが帯域に影響を与えないビデオビットレート要求の場合,そのユーザには利得低下のペナルティは与えず利得を保つことが保証される.既存研究\cite{kison}\cite{motomoto}では,利己的なユーザだけでなくリンクを共有する他ユーザにもペナルティを与える.これは表\ref{tb:ritoku1}を見れば明らかである.ユーザ1が6Mbpsを要求した場合,ユーザ1のみならずユーザ2も利得が減少する.しかし,提案利得関数ではユーザ2は利得が低下しない.それは表\ref{tab:teianritoku}を見れば明らかである.以下に表\ref{tab:teianritoku}を示す.
\begin{table}[h]
    \centering
    \bicaption{提案利得関数\ref{eq:f_buffi}の利得表}
                {Gain table for the proposed gain function\ref{eq:f_buffi}}
    \scalebox{0.7}{
    \begin{tabular}{|c|c|c|c|c|c|c|}
        \hline
        \diagbox{user1}{user2} & 1 & 2 & 3 & 4 & 5 & 6 \\ \hline
        1 & (0.5, 0.5) & (0.62, -1.0) & (0.67, -6.0) & (0.69, -16.0) & (0.7, -32.5) & (0.71, -57.0) \\ \hline
        2 & (-1.0, 0.62) & (0.0, 0.0) & (0.33, -2.62) & (0.5, -8.0) & (0.6, -16.88) & (0.67, -30.0) \\ \hline
        3 & (-6.0, 0.67) & (-2.62, 0.33) & (-1.5, -1.5) & (-0.94, -5.33) & (-0.6, -11.67) & (-0.37, -21.0) \\ \hline
        4 & (-16.0, 0.69) & (-8.0, 0.5) & (-5.33, -0.94) & (-4.0, -4.0) & (-3.2, -9.06) & (-2.67, -16.5) \\ \hline
        5 & (-32.5, 0.7) & (-16.88, 0.6) & (-11.67, -0.6) & (-9.06, -3.2) & (-7.5, -7.5) & (-6.46, -13.8) \\ \hline
        6 & (-57.0, 0.71) & (-30.0, 0.67) & (-21.0, -0.37) & (-16.5, -2.67) & (-13.8, -6.46) & (-12.0, -12.0) \\ \hline
    \end{tabular}
    \label{tab:teianritoku}
    }
\end{table}
これは既存研究\cite{kison}\cite{motomoto}の利得関数\ref{eq:f_{buffer}}では,バッファ変動量に基づいたシステム的なペナルティが設定されており,データ消費と貯蓄のバランスを重視している.一方で,提案利得関数\ref{eq:f_buffi}では,他ユーザとの比較を重視したペナルティ設計が行われており,この変更によりユーザの利得が相対的に評価されるようになった.この結果,帯域に影響を与えない低ビデオビットレート要求ユーザの利得が保たれ,高ビデオビットレート要求の利己的なユーザがより大きなペナルティを受ける仕組みが実現している.このように利得を制限することで,帯域に影響を与える高ビデオビットレートが最適レートとして導出されることを防ぐ.本研究では,ゲーム理論を用いてナッシュ均衡を導出することで,最適レートを導出している.ナッシュ均衡は各ユーザの利得の組み合わせがこれ以上大きくならないときのレートの組み合わせが選ばれる.そのため,利得を制限することで,そのレートの組み合わせが最適レートとして導出されないようにしている
しかし,利得が過剰に低下することで,ビデオビットレートも小さくなり,帯域利用の効率性やQoEの低下につながる可能性があるため,ペナルティのバランス調整が今後の課題となる.また,利得の下がり方がQoEにどのように影響するかについては,現状で明確な評価は行われていない.利得制限を反映させた初期段階であり,次のステップとして利得関数を基にしたレート制御を実現し,その上でQoE評価を行う必要がある.また,現時点ではユーザ数の増減に伴う利得の変化の解析が十分に検討されていない.ユーザ数が減少することで,一人当たりの帯域幅が増加し,高レート要求が選択されやすくなる一方,ユーザ数が増加すると,提案手法が帯域逼迫時にどのような挙動を示すかについての詳細な分析が必要である.

ゲーム理論を活用する意義についても議論があった.本研究ではゲーム理論を用いることで,利己的なユーザ間の相互作用をモデル化し,ナッシュ均衡解を導出している.単なる最適化やユーザの数で帯域を均等に割る手法では,ユーザ間の相互作用を考慮できないため,帯域幅の効率的な利用や公平性を担保することが難しい.本研究では,現時点でユーザ毎の特徴を考慮していない.Wangら\cite{C.Wang}は,ユーザの行動などの特性が動画視聴のQoEに影響を及ぼすことを示唆している.例えば,ラジオ感覚で動画を流し見するユーザや動画を高画質で視聴するユーザもいる.このようなユーザはそれぞれ要求するビデオビットレートが異なるため,帯域をユーザ数で等分する手法では,ユーザの公平性が担保されない.ゲーム理論を用いることで,非協力的なユーザが存在する状況下でも安定的な最適レートを導出することが可能である.しかし,現在ユーザ毎の行動特性を考慮していないため,今後特性を取り入れた関数の作成が検討される.また現状では,ユーザ全体の利得合計が既存手法に比べて低下する傾向が見られる.この点について,QoEを向上させるための調整が必要である.利得が極端に低い値を取る場合,帯域幅に余裕があるにも関わらず低レートしか選択されない状況が生じる可能性があるため,これを防ぐためのさらなる改良が求められる.

今後の研究では,以下の方向性を重点的に進める必要がある.まず,視聴動画の種類や画質要求の違いなど,ユーザ特性を利得関数に組み込むことで,より個別化されたレート制御を実現することが求められる.また,現在は利得の変化に基づいた評価が中心であるが,最終的にはQoEの向上を実証する必要がある.具体的には,提案手法を用いた場合のユーザ満足度の向上を定量的に評価する必要がある.さらに,現行のペナルティ項の構造を再評価し,よりシンプルかつ効果的な設計を模索する必要がある.加えて,現状では$2\sim4$人のユーザを対象としているが,多人数ユーザ環境における提案手法の有効性を検証することも重要である.

以上の方向性に基づき,提案手法の改善と実用性の向上を図ることで,QoE向上を目指したレート制御法の考案を今後検討する.