\chapter{結論}
本研究では,ユーザの利己的なビデオビットレート要求が他ユーザの使用帯域に与える影響を考慮した利得関数への改良を提案した.この利得関数により,帯域に影響を与えない低ビットレートを要求するユーザに対して過渡な制限を課すことなく,各ユーザの影響度に応じた効率的な帯域割り当ての実現性を示した.また,既存研究\cite{kison}と比較し,貯蓄されるバッファ量が多いため,動画の再生停止が起きにくいことが示された.
しかし,本研究には以下の課題が残されている.第一に,本研究では利得関数の利得制限の改良を行ったのみである為,実際の動画再生中における持続的かつ動的なレート制御が十分に実現されていない.特に,帯域の急激な変動やリンクへの複数ユーザの離脱や加入が発生した場合の影響を,時間的な変化に沿って解析することが求められる.
第二に,提案手法の評価ではバッファ量や利得を中心とした分析を行ったが,ユーザのQoEについての評価まで解析できていない課題がある.既存研究\cite{kison}では導出された最適レートを基にユーザごとの平均QoEを算出し評価しているため,QoEに関する比較分析が不十分である.
今後は,提案した利得関数の様々の条件でのQoE解析や関数自体の改善を検討する.