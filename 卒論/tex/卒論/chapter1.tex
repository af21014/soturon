\chapter{序論}
近年,ストリーミングサービスの普及と需要の増加に伴い,ネットワークトラヒックが増加し,インターネットトラヒックの中で動画が大きな割合を占めている.このトラヒックの増加により,安定して高画質な動画を配信することが重要な課題となっている\cite{kison}\cite{motomoto}.このような課題に対して,サーバとユーザ間で動画データを適応的に送信する技術として,アダプティブビットレートストリーミング(ABS)が活用されている.しかし,動画に関するトラヒックが大きく占め,限られているネットワーク帯域幅は,ユーザ数や通信量の変動,無線通信環境の特性,TCPの輻輳制御などの要因によって動的に変化する\cite{C.Wang}.このような課題は,ABSにおいて,ユーザの利己的な行動が帯域幅の利用効率低下や,ユーザ間の不公平なリソース配分によるユーザの動画再生中断につながり,対処しきれない\cite{kison}\cite{C.Wang}.この問題を解決するために,ユーザのビデオビットレートを制御し,安定して高画質な動画配信を実現するビデオビットレート制御の研究が進められている\cite{kison}\cite{motomoto}\cite{Thang}\cite{Mao}\cite{Xu}.

特に,複数のユーザが共有するリンクがボトルネックとなる場合,リンクの帯域幅が逼迫し,あるユーザの要求するビデオビットレートが他のユーザの使用帯域幅に影響を与える.結果として,共有しているユーザは希望するビデオビットレートのデータが十分に送信されない,あるいは動画の再生が停止することがある.このような課題を解決するためには,ユーザ間の相互影響を考慮した適切なビデオビットレート制御が必要である.
また,近年,ビデオコンテンツの品質に対するユーザ満足度を表す体験の質(QoE)に焦点を当てた方法が,サービスプロバイダーおよびネットワークプロバイダーの注目を集めている\cite{kison}.特に,HTTPモバイルストリーミングにおけるQoE評価手法に関する研究では,ユーザの行動パターンや動画視聴環境がQoEに与える影響が示されている\cite{Y.Huang}.同研究では,QoEの主な要因として,視聴動画の流暢性,つまり動画の再生停止が大きな影響を与えることが指摘されている.このため,動画の再生停止の原因であるバッファの枯渇の起きない適応的なビットレート制御が求められる.

本研究では,ユーザのビデオビットレートが他のユーザの帯域使用に及ぼす相互依存関係を,ゲーム理論を用いてモデル化する.ゲーム理論は,複数のユーザが共有資源を取り合う際に生じる相互依存関係を数理モデルを通じて解析する理論である\cite{okada2011}.本研究では,このゲーム理論を基盤に,ユーザの戦略となる要求ビデオビットレートに応じてユーザ毎に異なるペナルティを導入し,ユーザの要求ビデオビットレートが他ユーザの帯域使用に及ぼす影響を考慮した新たな利得関数を提案する.この提案利得関数を用いることにより,低ビットレートユーザに過渡な制限を与えず,各ユーザのビデオビットレート要求に応じた帯域割り当ての実現可能性を示す.さらに,この提案手法では既存研究\cite{kison}のビデオビットレート制御に比べ,バッファの貯蓄量が多く,動画の再生停止が起きにくいことを示す.

本論文の構成は以下の通りである.2章にて関連研究とその問題点について述べ,3章3,4節にて既存研究におけるゲーム理論を用いたレート制御法の説明を行う.3章5節にて提案するレート制御手法を説明をし,4章にて数値解析とその結果を述べる.5章にてまとめと本研究における問題点を述べる.


%%リンク共有時の動画再生停止を防ぐための既存の画質レート制御関数では,自身の要求画質による他ユーザの転送レートへの影響が考慮されていない.そこで本研究では,このような他ユーザの転送レートへの影響を考慮した画質レート制御関数を提案する.
